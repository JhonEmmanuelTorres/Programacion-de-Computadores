\documentclass[10pt,a4paper]{article}
\usepackage[utf8]{inputenc}
\usepackage{amsmath}
\usepackage{amsfonts}
\usepackage{amssymb}
\usepackage{makeidx}
\usepackage{graphicx}
\title{Informacion Estadistica}
\date{}
\author{Jhon Emmanuel Torres}
\begin{document}
\maketitle
\section{Objetivo}
Dar a conocer los principales conceptos de la estadística para su posterior aplicacion, mostrando por cada uno ejemplos para que sea mas claro para el lector

\section{Conceptos}
Sean los dos conjuntos\\\\
$
A = \left\lbrace  1, 2, 3, 4, 5, 5 \right\rbrace \\
B = \left\lbrace  6, 7, 8, 9, 9, 10, 10 \right\rbrace  
$

\subsection{Media Aritmética}
Sea $x_{1}, \ldots,x_{n}$ una muestra. La \textbf{media aritmética} es

$$ \overline{X} = \dfrac{1}{n} \sum_{i = n}^{n}x_{i} $$
\begin{flushleft}
\textbf{Ejemplo:}
\end{flushleft}
$
\overline{X}_{A} = 3.5\\
\overline{X}_{B} = 9
$

\subsection{Media Geométrica}

Sea $x_{1}, \ldots,x_{n}$ una muestra. La \textbf{media geométrica} es

$$ \overline{x} = \sqrt[n]{\prod_{i = 1}^{n} x_i } = \sqrt[n]{x_1\cdot x_2 \cdots x_n }
 $$

\begin{flushleft}
	\textbf{Ejemplo:}
\end{flushleft}
$
\overline{x}_{A} = 2.90\\
\overline{x}_{B} = 6.06
$



\subsection{Media Armónica}

Dados $n$ numeros $x_{1}, \ldots,x_{n}$. La \textbf{media armónica} es

$$ H = \dfrac{n}{ \dfrac{1}{x_1} + \cdots + \dfrac{1}{x_n} } =  \dfrac{n}{ \sum_{i = 1}^{n} \dfrac{1}{x_i} }
$$

\begin{flushleft}
	\textbf{Ejemplo:}
\end{flushleft}
$
H_{A} = 2.41\\
H_{B} = 8.17
$

\subsection{Mediana}

Sean $x_{1}, \ldots,x_{n}$ en un orden creciente. La media $M_e$ se define:
\begin{itemize}
	\item Sí $n$ es impar: es el valor que ocupa la posición $ M_e = \frac{n+1}{2}$.
	\begin{flushleft}
		\textbf{Ejemplo:}
	\end{flushleft}
	$ {M_e}_B = 9 $
	\item Sí $n$ es par: es el promedio de los valores centrales, por lo tanto se define como $ M_e = \dfrac{x_{\frac{n}{2}} + x_{\frac{n}{2}+1} }{2} $
	\begin{flushleft}
		\textbf{Ejemplo:}
	\end{flushleft}
	$ {M_e}_A = \dfrac{3 + 4}{2} = 3.5 $
\end{itemize}

\subsection{Moda}

Es el valor con mayor frecuencia en una distribución de datos. Para nuestra suite si hay varias datos con la misma frecuencia vamos a tomar el mayor.

\begin{flushleft}
	\textbf{Ejemplo:}
\end{flushleft}
$
M_A = 5\\
M_B = 10
$

\subsection{Varianza}

La varianza es una medida de dispersión y se define como

$$ \sigma^2 = \dfrac{1}{n} \sum_{i = 1}^{n} \left(  X_i - \overline{X} \right) ^2 $$
\begin{flushleft}
	\textbf{Ejemplo:}
\end{flushleft}
$
{\sigma_{A}}^2 = 1.49\\
{\sigma_{B}}^2 = 1.39
$

\subsection{Desviación Estándar}

La Desviación Estándar es una medida de dispersión y se define como la raíz cuadrada de la varianza

$$ \sigma = \sqrt{ \dfrac{1}{n} \sum_{i = 1}^{n} \left(  X_i - \overline{X} \right) ^2  }$$
\begin{flushleft}
	\textbf{Ejemplo:}
\end{flushleft}
$
{\sigma_{A}} = 1.63\\
{\sigma_{B}} = 1.51
$

\subsection{Cuartiles}

Los cuartiles son tres valores que dividen un conjunto de datos ordenados cuatro partes porcentualmente iguales. 

Sean $x_{1}, \ldots,x_{n}$ en un orden creciente. Se define los cuartiles como:
\begin{itemize}
	\item Primer Cuartil $(Q_1)$ es la mediana de la primera mitad de valores, por lo tanto es el dato que esta en la posición $\dfrac{n + 1}{4}$.
	\item Segundo Cuartil $(Q_1)$ es la mediana de los datos
		\item Primer Cuartil $(Q_1)$ es la mediana de la segundo mitad de valores, por lo tanto es el dato que esta en la posición $\dfrac{3(n + 1)}{4}$.
\end{itemize}


\subsection{Rango Intercuartílico}

El Rango Intercuartílico es la diferencia entre el tercer cuartil y el primer cuartil de los datos.
$$R = Q_3 - Q_1$$

\subsection{Asimetría de Pearson}
 Se define como
 
$$ A_p = \dfrac{ \overline{X} - M }{ \sigma } $$


\subsection{Asimetría de Bowley}
Se define como

$$ A_p = \dfrac{ Q_3 + Q_1 - 2M_e }{ Q_3 - Q_1 } $$

\subsection{Permutacion}

Sea $n$ el numero de elementos en el conjunto y $r$ numero de elementos que se desean, entonces las diferentes formas que podemos escoger $r$ elementos teniendo en cuenta el orden es

$$ {P_r}^n = \dfrac{n!}{ \left( n-r \right)! } $$

\subsection{Combinatoria}

Sea $n$ el numero de elementos en el conjunto y $r$ numero de elementos que se desean, entonces las diferentes formas que podemos escoger $r$ elementos sin tener en cuenta el orden es


$$ {C_r}^n = \binom{n}{r} = \dfrac{n!}{ \left( n-r \right)! r! } $$


\end{document}
