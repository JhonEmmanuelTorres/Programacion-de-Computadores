\documentclass[10pt,a4paper]{article}
\usepackage[utf8]{inputenc}
\usepackage{amsmath}
\usepackage{amsfonts}
\usepackage{amssymb}
\usepackage{makeidx}
\usepackage{graphicx}
\usepackage{hyperref}
\title{Programación de Computadores}
\date{}
\begin{document}
\maketitle
\section{Objetivo}
Construir una suite estadística en la cual podamos utilizar los principales conceptos de la programación estructurada por medio del lenguaje de programación C++, en el cual se cree una interfaz por medio de la terminal que sea amigable con el usuario final.
\section{Descripcion:}


\begin{itemize}
	\item El programa debe mostrar una pantalla de bienvenida.
	\item En el menú principal deben estar la siguientes opciones:
		
	\begin{enumerate}
		\item Cargar datos
		\item Análisis Estadístico
		\item Combinatoria
		\item Acerca de
		\item Salir
	\end{enumerate}

	\item En Análisis Estadístico debemos hacer otro menú con la siguientes opciones:
	\begin{enumerate}
		\item Máximo valor
		\item Mínimo valor
		\item Rango
		\item Media
		\item Media Geométrica
		\item Medio Armónica
		\item Mediana
		\item Moda
		\item Varianza
		\item Desviación Estándar
		\item Cuartil
		\item Rango Intercuartilico
		\item Asimetria de Pearson
		\item Asimetria de Bowley
		\item Asimetria de Fisher
		\item Coeficiente de Apuntamiento		
	\end{enumerate}
	
	\item En Combinatoria debemos hacer otro menú con las siguientes opciones:
	\begin{enumerate}
		\item Permutacion
		\item Combinatoria
	\end{enumerate}
	
	\item En "Acerca de" debemos hacer un pantallazo mostrando la información del programa.
	
	\item En la opción "Salir" debemos hacer un menú preguntándole al usuario si realmente esta seguro de salir.
	
\end{itemize}

\section{Consideraciones:}

\begin{itemize}
	\item El proyecto debe hacerse en el sistema operativo Windows.
	\item Es muy útil organizar el nuestros datos para poder hacer nuestro analisis estadistico esto lo logramos de la siguiente manera.
	\begin{enumerate}
		\item Importamos la siguiente librería:
			\begin{verbatim}
			#include <algorithm> 
			\end{verbatim}
		\item Ordenamos con la siguiente instrucción:
			\begin{verbatim}
			sort( x, x+n );
			\end{verbatim}
	\end{enumerate}
	\item Las firmas de las función deben ser la siguientes:
	\begin{verbatim} 
		* void cargarDatos( string ruta, double *x, int &n );
		* double max( double *x, int n );
		* double min( double* x, int n );
		* double rango( double* x, int n );
		* double media( double* x, int n );
		* double mediaGeometrica( double* x, int n );
		* double mediaArmonica( double* x, int n );
		* double mediana( double* x, int n );
		* double moda( double* x, int n );
		* double varianza( double* x, int n );
		* double desviacionEstandar( double* x, int n );
		* double rangoIntercuartilico( double* x, int n );
		* double asimetriaPearson( double* x, int n );
		* double asimetriaBowley( double* x, int n );
		* double asimetriaFisher( double* x, int n );
		* double coeficienteApuntamiento( double* x, int n );
		* int permutacion( int n, int k );
		* int combinacion( int n, int k );
	\end{verbatim}
	
	\item Hay algunas funciones importantes en la librería cmath. \href{http://www.cplusplus.com/reference/cmath/}{Aquí}
	 esta la descripción de todas las funciones en esta librería.

	\item Para cargar los datos al sistema es útil la librería fstream. \href{http://www.cplusplus.com/reference/fstream/}{Aquí}
	esta la descripción.
	\item El formato del archivo que vamos a cargar es el siguiente: primero colocamos numero de datos seguidos de nuestros datos a analizar todos estos separados por un espacio.
	\begin{flushleft}
		\text{Ejemplo:}
	\end{flushleft}
	\begin{verbatim}
	4 5.6 86.8 6.4 5.6
	\end{verbatim}
	\item Siempre debemos limpiar la memoria reservada con el comando \begin{verbatim}delete\end{verbatim}
	\item Es posible cambiar los colores de la consola
\end{itemize}
	
\end{document}