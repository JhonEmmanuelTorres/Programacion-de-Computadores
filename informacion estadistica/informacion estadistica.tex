\documentclass[10pt,a4paper]{article}
\usepackage[utf8]{inputenc}
\usepackage{amsmath}
\usepackage{amsfonts}
\usepackage{amssymb}
\usepackage{makeidx}
\usepackage{graphicx}
\title{Informacion Estadistica}
\date{}
\author{Jhon Emmanuel Torres}
\begin{document}
\maketitle
\section{Objetivo}
Dar a conocer los principales conceptos de la estadística para su posterior aplicacion, mostrando por cada uno ejemplos para que sea mas claro para el lector

\section{Conceptos}
Sean los dos conjuntos\\\\
$
A = \left\lbrace  1, 2, 3, 4, 5, 5 \right\rbrace \\
B = \left\lbrace  6, 7, 8, 9, 9, 10, 10 \right\rbrace  
$

\subsection{Media Aritmética}
Sea $x_{1}, \ldots,x_{n}$ una muestra. La \textbf{media aritmética} es

$$ \overline{X} = \dfrac{1}{n} \sum_{i = n}^{n}x_{i} $$
\begin{flushleft}
\textbf{Ejemplo:}
\end{flushleft}
$
\overline{X}_{A} = 3.5\\
\overline{X}_{B} = 9
$

\subsection{Media Geométrica}

Sea $x_{1}, \ldots,x_{n}$ una muestra. La \textbf{media aritmética} es

$$ \overline{x} = \sqrt[n]{\prod_{i = 1}^{n} x_i } = \sqrt[n]{x_1\cdot x_2 \cdots x_n }
 $$

\begin{flushleft}
	\textbf{Ejemplo:}
\end{flushleft}
$
\overline{X}_{A} = 3.5\\
\overline{X}_{B} = 9
$











\end{document}
